%******************************************%
%                                          %
% Template standard per la Tesi di Laurea  %
%            di Stefano Bianchi            %
%                                          %
%         versione: 27 marzo 2014          %
%                                          %
%******************************************%

% Replace oneside with twoside for twoside printing
\documentclass[a4paper,11pt,oneside,openany]{book}

\usepackage[english]{babel}

\usepackage{fancyhdr} % Cover rendering

\usepackage{graphicx} % Enables usage of figures, including subfigures with subcaptions
\usepackage{caption}
\usepackage{subcaption}

\usepackage{tabularx} % Extra functionality for tables
\usepackage{longtable} % Enables writing tables on multiple pages

\usepackage[titletoc]{appendix} % Enables adding the appendix chapters to the index

\usepackage{float} % Used for floating images

\usepackage{amsmath} % Enables usage of math formulas
\numberwithin{equation}{section}

\usepackage{listings} % Consente la formattazione di codice sorgente

\usepackage{multirow} % Enables merging of columns in tables

% Using biblatex with biber backend for bibliography management
% Provides much more customization options than bibtex
\usepackage{csquotes}
\usepackage[backend=biber,style=numeric,sorting=none]{biblatex}
\addbibresource{bib/bibliography.bib}

\usepackage{hyperref} % Enables clickable hyperreferences

\usepackage{lipsum} % Used only for the placeholder Lorem ipsum text, can be removed

% Your content should be added in the files referenced by the \input{ } commands

%*******************************************************
% Page and cover settings
%*******************************************************
% Clickable hyperlinks settings
\hypersetup{
    colorlinks,
    citecolor=green,
    filecolor=cyan,
    linkcolor=red,
    urlcolor=magenta
}

% Headers' settings
\fancyhf{}
\pagestyle{fancy}
\lhead{}
\chead{\leftmark}
\rhead{}
\lfoot{}
\cfoot{}
\rfoot{\thepage}

% Set the root directory for images
\graphicspath{img}

% style settings for multi-line and inline code
% see this comment https://tex.stackexchange.com/a/395833 for more details and additional customization
\lstset{
    showstringspaces=false,
    basicstyle=\ttfamily,
    keywordstyle=\color{blue},
    commentstyle=\color[gray]{0.6},
    stringstyle=\color[RGB]{255,150,75}
}


%*******************************************************
% Cover
%*******************************************************
\usepackage[some]{background}
\usepackage{amsfonts}
\SetBgScale{1}
\SetBgContents{
    % Set the appropriate logo here
    \includegraphics[scale=1]{img/units_sigillo_pantone-534}}
\SetBgColor{gray}
\SetBgAngle{0}
\SetBgOpacity{0.07}

\usepackage{textcomp}
\usepackage{color}\title{Thesis title}
\author{Name Surname}
\begin{document}

    \pagenumbering{roman}
\begin{titlepage}
    \begin{center}
        {\LARGE {\bfseries UNIVERSITÀ DEGLI STUDI DI TRIESTE \\}}
        \vspace{.5cm}
        {\Large {\bfseries Dipartimento di Ingegneria e Architettura \\}}
        \vspace{1cm}
        \includegraphics[width=4cm,height=4cm]{img/units_sigillo_pantone-534}\\[1.5cm]

        {\LARGE
            Laurea Magistrale in Ingegneria Elettronica e Informatica \\
        }
        {\LARGE
                Applicazioni Informatiche \\
        }
        \vspace{1cm}
        {\LARGE
            {\bfseries Titolo tesi}
        }
        \vspace{1cm}

        % Set the correct date before printing the thesis
        {\large \today \\
        }

        \vfill
        \begin{table}[h]\label{tab:table}
            {\large
                    \begin{tabular}{c c c c r c c | c c l}
                        & & & & Studente & & & & & Relatore \\
                        & & & & \bfseries Nome Cognome & & & & & \bfseries Prof.\ Nome Cognome \\
                        & & & & & & & & & \\
                        & & & & & & & & & Correlatore \\
                        & & & & & & & & & \bfseries Prof.\ Nome Cognome \\
                    \end{tabular}
                }
        \end{table}
        Anno Accademico 2022/2023
    \end{center}
\end{titlepage}


%*******************************************************
% Dedication
%*******************************************************
    \phantomsection
\thispagestyle{empty}
\vspace*{3cm}

\begin{center}
    Lorem ipsum \\
    dolor sit amet. \\
    --- Cicero -- \emph{De finibus bonorum et malorum} ---
\end{center}

\medskip

\begin{center}
    \hfill \emph{Ad maiora!} \\
\end{center}


    \frontmatter

%*******************************************************
% Abstract
%*******************************************************
    \chapter{Sommario}\label{ch:sommario}

\lipsum[1]


%*******************************************************
% Table of Contents
%*******************************************************
    {\hypersetup{linkcolor=black}
        \tableofcontents
    }

%*******************************************************
% Introduction
%*******************************************************
    \renewcommand{\chaptermark}[1]{\markboth{\MakeUppercase{\ #1}}{}}

    \chapter{Introduzione}\label{ch:introduzione}

\lipsum


%*******************************************************
% Chapters
%*******************************************************
    \mainmatter

    \renewcommand{\chaptermark}[1]{\markboth{\MakeUppercase{\chaptername\ \thechapter.\ #1}}{}}

    % You can add more chapters here
    % Each chapter is defined in a separate source file

    \chapter{Lorem}\label{ch:lorem}

\section{Ipsum}\label{sec:ipsum}

    Citation sample~\cite{latexcompanion}.

    \begin{figure}[!ht]\label{fig:example}
        \centering
        \includegraphics[width=.8\textwiimg]{img/units_sigillo_pantone-534}
        \caption{A figure}
    \end{figure}

    \lipsum[3]

    \begin{figure}[!ht]
        \centering
        \subfigure[Subfigure 1]{\includegraphics[width=.45\textwiimg{img/units_sigillo_pantone-534}}
        \subfigure[Subfigure 2]{\includegraphics[width=.45\textwidth]{img/units_sigillo_pantone-534}}
        \subfigure[Subfigure 3]{\includegraphics[width=.45\textwidth]{img/units_sigillo_pantone-534}}
        \subfigure[Subfigure 4]{\includegraphics[width=.45\textwidth]{img/units_sigillo_pantone-534}}
        \caption{A figure with subfigures}
        \label{fig:subexample}
    \end{figure}

    \lipsum[4]

\section{Dolor}\label{sec:dolor}

    \lipsum[3]

    \begin{table}
        \centering
        \begin{tabular}{|c|c|}
            \hline
            \textbf{Lorem} & \textbf{Ipsum} \\
            \hline
            \hline
            Dolor & Sit \\
            \hline
            Amet &  \\
            \hline
        \end{tabular}
        \caption{A table}
        \label{tab:example}
    \end{table}

    \lipsum[4-5]

    \chapter{Sit Amet}\label{ch:sit-amet}

\lipsum[1]

\section{Lorem Ipsum}\label{sec:lorem-ipsum}

\lipsum[2-4]

    \subsection{Dolor sit amet}\label{subsec:dolor-sit-amet}

        \lipsum[5-7]
    \chapter{Lorem Ipsum}\label{ch:lorem-ipsum}

\lipsum[1]

\section{Dolor}\label{sec:dolor2}

    \lipsum[2-4]

    \subsection{Sit amet}\label{subsec:sit-amet}

        \lipsum[5-12]
    \chapter{Dolor}\label{ch:dolor}

\lipsum[1]

\section{Lorem Ipsum}\label{sec:lorem-ipsum2}

    \lipsum[2-4]

    \subsection{Dolor sit amet}\label{subsec:dolor-sit-amet2}

        \lipsum[5-11]

%*******************************************************
% Conclusion
%*******************************************************
    \chapter*{Conclusion}

% Do not edit these lines
\addcontentsline{toc}{chapter}{Conclusion}
\label{chap:concl}
\markboth{CONCLUSION}{CONCLUSION}
% Do not edit these lines

\lipsum[1]

\section{Lorem Ipsum}

    \lipsum[2-4]

    \subsection{Dolor sit amet}

        \lipsum[5-7]


%*******************************************************
% Appendix
%*******************************************************
% Do not edit this part
    \pagestyle{fancy}
    \renewcommand{\chaptermark}[1]{\markboth{\MakeUppercase{APPENDICE\ \thechapter.\ #1}}{}}
% Do not edit this part

    \appendix
    \label{appendix}

    \input{back/appendix1.tex}
    \input{back/appendix2.tex}
    \chapter{Lorem}

\lipsum[1]

\section{Lorem Ipsum}

    \lipsum[2-4]

    \subsection{Dolor sit amet}

        \lipsum[5-7]

    \backmatter

%*******************************************************
% Bibliografia
%*******************************************************~\nocite{*}
    \printbibliography

%*******************************************************
% Dedica finale o ringraziamenti
%*******************************************************
    % Do not edit
\chapter*{}
\thispagestyle{empty}
\vspace*{3cm}
% Do not edit

\begin{center}
    \hfill Ei fu.\ Siccome immobile, \\
    \hfill Dato il mortal sospiro, \\
    \hfill Stette la spoglia immemore \\
    \hfill Orba di tanto spiro \\ \medskip
    \hfill Alessandro Manzoni -- \emph{Il Cinque Maggio}
\end{center}

\end{document}
